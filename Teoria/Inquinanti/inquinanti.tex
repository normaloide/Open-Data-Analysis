\documentclass[12pt, letterpaper]{article}
\usepackage[utf8]{inputenc}
\usepackage{parskip}

\title{Inquinanti analizzati}
\author{Open Data Analysis}
\date{Maggio 2020}

\begin{document}

\begin{titlepage}
	\maketitle
	\begin{abstract}
	In questo documento vengono illustrati i vari inquinanti analizzati all'interno del 		progetto. In particolare il testo cerca di offrire una conoscenza preliminare e non 		troppo specifica, su come le sostanze inquinanti vengono prodotte e immesse nella 			nostra atmosfera e su come esse interagiscono tra loro e sull'ambiente.
	\end{abstract}
\end{titlepage}

\section{$\textbf{PM}_{10}$}
Il \textit{Particulate Matter 10} o \textit{Materia Particolata 10} è composto da aerosol, ovvero da un insieme di particolati quali polvere, fumo, fuliggine e nebbia. Il numero 10 è dovuto al fatto che tutti questi particolati hanno diametro inferiore ai 10µm. Circa il 60\% del $\textrm{PM}_{10}$ è composto da $\textrm{PM}_{2,5}$, argomento della sezione successiva.

L'origine del $\textrm{PM}_{10}$ è da attribuirsi sia a fattori di origine naturale, sia a cause antropologiche: sorgenti geologiche come l'erosione del suolo, gli incendi e il vulcanismo contribuiscono al bilancio annuo di materia particalata, a queste fonti vanno aggiunte quelle che derivano dall'attività umana, come per esempio i processi di combustione sia di motori a scoppio, sia dovuti agli impianti di riscaldamento.
Sempre dovuta all'attività dell'uomo, e in particolare alle industrie metallurgiche e chimiche, è l'emissione di sostanze quali azoto e zolfo, che tramite condensazione vanno a comporre il $\textrm{PM}_{10}$.

Il particolato 10µm è una tra le cause di asme e bronchiti, e se delle dimensioni del $\textrm{PM}_{2,5}$ diventa successivamente causa di malattie cardio-respiratorie e tumori.

\section{$\textbf{PM}_{2,5}$}
Il \textit{Particulate Matter 2,5} come citato nella sezione precedente, è un particolato che compone il 60\% del $\textrm{PM}_{10}$. Esso è chiamato 2,5 o 25 proprio per le sue dimensioni, inferiori ai 2,5µm.

A causa delle dimensioni così ridotte, riesce a raggiungere i bronchi dell'uomo e a causare malattie di natura cancerogena.

\section{$\textbf{NO}_{2}$}
L'\textit{$\textrm{NO}_{2}$} ovvero il \textit{diossido di azoto}, si presenta come un gas rosso bruno e viene prodotto in natura da eruzioni vulcaniche e respirazione batterica. Il suo scopo, talvolta benefico, è di regolare reazioni chimiche che avvengono nella troposfera (soprattutto in zone dove vi è scarsità di ozono), dove appunto resta intrappolato a causa della sua densità maggiore di quella dell'aria.

Le sorgenti umane di diossido di azoto, derivano come nei casi precedenti, dall'energia fossile (motori, butano, cherosene,...), ma anche per esempio dal fumo delle sigarette, tuttavia non direttamente: la combustione fossile produce \textrm{NO}, che legandosi a composti organici volatici (\textrm{NO}) produce $\textrm{NO}_{2}$, che legandosi ulteriormente con altri composti volatili andrà a costituire altro particolato o acido nitrico, causa delle pioggie acide. Infine quando l'$\textrm{NO}_{2}$ viene colpito dalla luce solare, produce un ulteriore composto, l'$\textrm{O}_{3}$, ovvero l'ozono.

Il diossido di azoto è causa di convulsioni, dolori toracici e problemi di insufficienza circolatoria.

\section{$\textbf{SO}_{2}$}
Il \textit{diossido di zolfo} anche noto con il nome di \textit{anidride solforosa}, è un gas incolore e naturalmente immesso nell'atmosfera terrestre tramite attività vulcanica. Le principali cause umane che portano all'emissione di questo inquinante sono i centri di lavorazione del rame e, ancora una volta, i combustibili fossili. Da circa 4 decenni, l'inquinamento da $\textrm{SO}_{2}$ è diminuito notevolmente grazie all'utilizzo sempre crescente del metano, tuttavia restano ancora un rischio le centrali termoelettriche che però attuano dei processi di desolforazione.

Le principali problematiche che insorgono con livelli eccessivi di $\textrm{SO}_{2}$ riguardano congiuntiviti e faringiti, in quanto il gas risulta essere un potente irritante. Infine il diossido di zolfo reagisce con il diossido di azoto andando a formare $\textrm{SO}_{3}$ e \textrm{NO}.

\section{CO\_8h}
Con questa nomenclatura si indica la concentrazione media di \textrm{CO} su 8 ore. Il \textit{CO} meglio noto come \textit{monossido di carbonio}, è un gas incolore, insapore e inodore. La sua presenza nell'aria può essere dovuta da incendi boschivi o vulcanismo, ma per la maggior parte da reazioni fotochimiche al livello della troposfere, mentre è possibile generare \textrm{CO} tramite combustione in luoghi chiusi, come per esempio quelle che avvengono nelle stufe.

Il monossido di carbonio può portare alla morte per ipossia, scaturita dal legame che avviene tra \textrm{CO} e lo ione di ferro nell'emoglobina, molto più forte di quello ossigeno, impedendo quindi il trasporto di ossigeno nel sangue.

\section{$\textbf{O}_{3}$}
$\textit{O}_{3}$ è la formula chimica utilizzata per identificare l'ozono. Esso è da classificarsi come un gas serra in quanto assorbe e trattiene l'energia del Sole. Se l'ozono viene a trovarsi nella troposfera, risulta essere un inquinante estremamente nocivo se respirato in grandi dosi. Come citato sopra, l'ozono può venire a formarsi tramite reazioni chimiche del diossido di azoto e viene impiegato principalmente come sbiancatore e disinfettante nelle industrie alimentari, plastiche e negli acquedotti.

In dosi elevate può provocare asma, dispnea (problemi respiratori) e rallentamento della fagocitosi, ovvero la capacità delle cellule di ingerire materiali estranei, compresi patogeni.

\section{$\textbf{C}_{6}\textbf{H}_{6}$}
Con $\textit{C}_{6}\textit{H}_{6}$ viene indicato il \textit{benzene}, un liquido voltaile incolore prodotto naturalmente da eventi ignifughi, quali incendi boschivi o vulcanismo, oppure prodotto dall'uomo tramite combustioni domestiche (stufe, camini,...) e non, fumo di sigaretta ed è possibile trovarlo in alte concentrazioni nelle officine, in luoghi molto trafficati o con presenza di vernici e adesivi freschi, come ad esempio industrie per tali prodotti o in semplici cantieri di ristrutturazione o rifinitura di edifici.

Questo elemento chimico causa da disturbi lievi quali tremori, vertigini, tachicardia, sonnolenza, a disturbi più gravi quali immunodepressione e problemi coagulatori. Inoltre è stata evidenziata una correlazione tra luoghi di lavoro con alto utilizzo di vernici e gomme e un alto tasso di malati di anemia aplastica e leucemia mieloide acuta.

\end{document}